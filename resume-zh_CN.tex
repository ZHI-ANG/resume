% !TEX TS-program = xelatex
% !TEX encoding = UTF-8 Unicode
% !Mode:: "TeX:UTF-8"

\documentclass{resume}
\usepackage{zh_CN-Adobefonts_external} % Simplified Chinese Support using external fonts (./fonts/zh_CN-Adobe/)
%\usepackage{zh_CN-Adobefonts_internal} % Simplified Chinese Support using system fonts
\usepackage{linespacing_fix} % disable extra space before next section
\usepackage{cite}

\begin{document}
\pagenumbering{gobble} % suppress displaying page number

\name{李志昂}

% {E-mail}{mobilephone}{homepage}
% be careful of _ in emaill address
\contactInfo{(+86) 130-1695-9399}{zhiang-li@outlook.com}{}{}
% {E-mail}{mobilephone}
% keep the last empty braces!
%\contactInfo{xxx@yuanbin.me}{(+86) 131-221-87xxx}{}
 
%\section{个人总结}
%本人在校成绩优秀、乐观向上,工作负责、自我驱动力强、热爱尝试新事物,认同开放、连接、共享的Web在未来的不可替代性。在校期间长期从事可视分析(Web的2D/3D时空可视化)相关研究,对Web技术发展趋势及前端工程化解决方案有浓厚兴趣。\textbf{现任职于阿里巴巴集团。}

% \section{\faGraduationCap\ 教育背景}
\section{教育背景}
\datedsubsection{\textbf{新加坡国立大学},计算机科学,\textit{在读硕士研究生}}{2021.01 - 至今}
\begin{itemize}[parsep=0.2ex]
  \item \textbf{GPA: 4.5/5.0},预计2022年12月毕业
\end{itemize}
\datedsubsection{\textbf{东南大学},电子科学与工程,\textit{工学学士}}{2015.09 - 2019.06}
\begin{itemize}[parsep=0.2ex]
  \item \textbf{GPA: 3.6/4.0 | 排名6/62(前10\%)}
  \item \textbf{获奖荣誉}:东南大学校长奖学金,全国大学生电子设计竞赛一等奖,江苏省大学生创新创业优秀项目
\end{itemize}

% - 编程语言:C/C++, Python, OCaml, Scala, Java
% - 开发环境与工具:Linux, CMake, Makefile, git, GDB, Clang, QEMU, strace
% - 机器学习框架:Darknet, Caffe, OpenCV, PCL, Eigen
% - 嵌入式与机器人:STM32, Raspberry PI, NVIDIA TX2
% \section{\faCogs\ IT 技能}

% \end{itemize}

\section{实习经历}
\datedsubsection{\textbf{研究助理 @ Advanced Digital Sciences Center, Illinois SG, 新加坡}}{2022.05 - 至今}
\begin{itemize}
  \item \textbf{基于可信执行环境的工业控制设备安全架构设计}: 调研工业物联网场景下智能设备面临的威胁模型,利用可信计算技术ARM TrustZone和TEE保障设备的代码和数据安全。
  \item 编写可信应用,利用安全启动和安全存储特性,保障设备关键代码和私钥的完整性和保密性;交叉编译OP-TEE和WolfSSL,在TEE中实现了Modbus/TLS协议。
  \item 优化TEE中网络I/O的性能开销,借助qemu,gprof和strace等工具,通过实现缓冲区缓存网络数据包,降低了I/O读取次数;最终在树莓派3B上达到了平均50fps的处理速度。
\end{itemize}

\datedsubsection{\textbf{助理工程师 @ 清华大学苏州汽车研究院, 江苏苏州}}{2019.07 - 2019.10}
\begin{itemize}
  \item \textbf{激光雷达SLAM算法测试}:调研自动驾驶中激光雷达传感器定位和建图算法,协助研究员复现论文结果,统计路测误差并优化算法,形成技术文档并作技术分享。
  \item 重构算法库,采用C++11风格封装代码功能模块,规范数据通讯格式,兼容自动驾驶流水线。
  \item 自行实现一套功能完整的激光雷达建图算法,使用Ceres和PCL构建局部地图,ROS实现模块间通信,围绕园区一周建图闭环最大误差为5m,可以满足园区内对车辆定位的要求。
\end{itemize}


\section{项目经历}
\datedsubsection{\textbf{项目负责人 @ 东南大学电子学院, 江苏南京}}{2019.02 - 2019.05}
\begin{itemize}
  \item \textbf{高分辨率街景中的交通标志检测与识别系统}:在TT100K数据集2K分辨率街景图中检测和识别42种常见的交通标志。在标志尺寸存在样本数严重不均衡和尺寸跨度超过400倍的情况下,达到了90.7\%的mAP和93.8\%的召回率。
  \item 检测网络基于YOLO v3设计,调试锚框大小提升不同尺寸交通标志的检测率; 针对交通标志颜色通道较少的特点,裁剪卷积核数量;总体召回率达到94.7\%,参数量仅为YOLO v3的1/17。
  \item 识别网络采用全局平均池化,降低参数量,避免过拟合;量化模型权重,在不显著损失精度的前提下,在Pascal K80上达到了97.8\%的识别精度和超过1000fps的处理速度,模型仅为5.1KB。
  \item 采用数据增广进一步提升模型的检测精度。针对样本数量不均,以及遮挡,扭曲,反光,色差等问题,编写自动化脚本生成复杂样本,提升模型的泛化能力。最终获得了1.7\%的精度提升。
\end{itemize}

\datedsubsection{\textbf{项目负责人 @ 东南大学LabVIEW俱乐部, 江苏南京}}{2016.10 - 2017.04}
\begin{itemize}
  \item \textbf{投篮机器人目标检测系统}:识别并定位排球、球筐与障碍物;为捡球和投篮任务提供坐标。
  \item 设计了轻量级二维激光雷达点云滤波和分割算法,适应嵌入式平台算力,实现了噪点剔除和目标初筛,生成候选点云团簇,极大降低了后续检测算法的计算量。
  \item 采用最小二乘曲线拟合算法检测排球和投篮标定柱。对点云分割得到的每个点云团簇逐个拟合。在ARM Cortex A9平台上达到了半径5米内98.8\%的检测精度和超过15fps的检测速度。
  \item 该项目获得江苏省大学生创新创业优秀结题项目。
\end{itemize}

\section{技术能力}
% increase linespacing [parsep=0.5ex]
\begin{itemize}[parsep=0.2ex]
  \item \textbf{编程语言}: C/C++, Python, Scala, OCaml, Java
  \item \textbf{开发环境与工具}: Linux, CMake, Makefile, git, GDB, Clang, QEMU, strace
  \item \textbf{机器学习框架}: Darknet, Caffe, OpenCV, PCL, Eigen
  \item \textbf{嵌入式与机器人}: STM32, Raspberry PI, NVIDIA TX2
\end{itemize}

\end{document}
